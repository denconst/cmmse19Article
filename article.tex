
%Document generated using DScaffolding from https://www.mindmeister.com/1192580069
\documentclass{article}
\usepackage[utf8]{inputenc}

\newcommand{\todo}[1] {\iffalse #1 \fi} %Use \todo{} command for bringing ideas back to the mind map

\title{cmmse19}
\author{}

\begin{document}

\maketitle
      

\textbf{
A major problem is that solve large decision making  processes exactly on mobile robotic platforms. This problem is of particular concern as it can lead to execute decision making problems on low-power is slow and power consumming and curse of dimensionality. 
}\section{Introduction}

%Describe the practice in which the problem addressed appears

    
%Describe the practical problem addressed, its significance and its causes
A major problem is that solve large decision making  processes quickly and efficiently on mobile robotic platforms. This problem is of particular concern as it can lead to execute decision making problems on low-power is slow and power consumming and curse of dimensionality. 
    
%Summarise existing research including knowledge gaps and give an account for similar and/or alternative solutions to the problem

    
%Formulate goals and present Kernel theories used as a basis for the artefact design

    
%Describe the kind of artefact that is developed or evaluated
This article presents a novel artefact
    
%Formulate research questions

    
%Summarize the contributions and their significance
\textbf{
It is hoped that this research will contribute to a deeper understanding of the practice. We propose a solution aiming at complementing current approaches for solving solve large decision making  processes exactly on mobile robotic platforms. Consequently, this study can be classified as an improvement along Gregor and Hevner’s DSR knowledge contribution framework \cite{Gregor2013}.
}\textbf{
It is hoped that this research will contribute to a deeper understanding of the practice. We propose a solution aiming at complementing current approaches for solving solve large decision making  processes quickly and efficiently on mobile robotic platforms. Consequently, this study can be classified as an improvement along Gregor and Hevner’s DSR knowledge contribution framework \cite{Gregor2013}.
}
      
%Overview of the research strategies and methods used
This article has followed a Design Science Research approach.

%Describe the structure of the paper
The remainder of the paper is structured as follows: 

%Optional - illustrate the relevance and significance of the problem with an example
    
      
\bibliographystyle{unsrt}
\bibliography{references}

\end{document}
    